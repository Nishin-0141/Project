\chapter{結果}
\section{プロジェクトの結果}
プロジェクトの結果は、現段階で成果物が完成しておらず評価実験も実施していないため、記述することができない。今後の見通しとしては、ゲームの実装方法や種類について記述する予定である。
\bunseki{伊藤太一}

\section{成果の評価}
グループBの評価は、現段階で成果物が完成しておらずユーザやグループメンバから評価を受けることができない。今後の見通しとしては、ユーザからのレビューや最終成果発表での評価を記述する予定である。
\bunseki{中川翔真}


\newpage
\section{担当分担課題の評価}
%各人の担当課題の成果について、成果によってどのように上述した課題が解決されたか、 要求された役割は果たせたか、残された問題点はあるかを記述する。
\subsection{岩上慎之介}
はじめに、個人としてTeXとゲームの制作の2つを主に行ってきた。TeXでの執筆は初めてで、書き方をグループメンバに教わったり、Webで調べるなどして知識を増やした。グループ報告書制作では、まず章立てを行いそれに伴った執筆担当を決めていった。その後、作成した文章をTeXで書きグループ報告書を作成した。\\
 ゲーム開発班としては、ゲームのデモンストレーション段階の制作を課題としていた。現段階では、3つのゲームを制作した。3つのゲームの説明は以下に記述する。
1つ目は、Unityの基礎的な使い方を学ぶため難易度が低い「ブロック崩し」を制作した。Unityの性質やスクリプトとオブジェクトの紐づけに関して学ぶことができた。2つ目は、Unityでの配列の管理とマス目上でのオブジェクトの管理が、成果物を作るうえで良い知識になると考え「五目並べ」を制作した。マス目での入出力の仕方や座標を配列で管理する方法を学ぶことができた。3つ目は、実際にデモ画面を制作するということで、「ヒットアンドブロー」を制作した。配列の管理を学習していたこともあり、莫大な時間をかけず制作することができた。改善点としては、文字を出力する際のアニメーションや効果音の追加などが挙げられる。\\
 今後の課題としてUIの向上や他のゲームの実装、スマートフォンでのデモンストレーションなどが残っている。また、機械学習班で作成した辞書をUnityで使えるように引き継ぐのも重要な課題となっている。現段階で考えているのは、Python for .NETを使用しPythonからC\#にポインタ渡しを行う手段である。機械学習班では、辞書をPythonを使用して作成するため、PythonとC\#で配列を受け渡しする方法が有効である。先ほど記述した手段は、Python for .NETを使用することによりPythonとC\#を同一プロセスで動かすことができる。今後これらの課題を、ゲーム制作班と機械学習班と話し合いながら解決していくことを目標とする。
\bunseki{岩上慎之介}

\subsection{田中龍仁}
はじめに、グループBで私はゲーム開発班として活動してきた。中間発表案での主な課題は、ゲームのデモンストレーション作成、そして作成したゲームのデモンストレーションを動画として作成することである。グループBでは、9個のゲームを作成しミニゲーム集を作るという目的で活動してきた。まず、どのゲームをデモンストレーションとして作成するか、誰がゲームのデモンストレーションを作成するかを話し合った。その結果、投票結果の多い3個のゲームが候補に挙げられ、それらのゲームのデモンストレーションを作成することになった。私は、ゲーム制作に興味があったので今回は「クロスワードパズル」のデモンストレーション作成に立候補した。この「クロスワードパズル」は、通常のクロスワードパズルと違い、使われている言葉が北海道に関連したもので作られている。そして、ヒントをもとに、縦・横の文字数に合う言葉を埋めていき、盤面を完成させるというルールだ。デモンストレーション段階では各マスへの入力と盤面の作成を目標とした。今回のデモンストレーション作成は、Processingで行った。一番苦労したところは、Processing上でのキーボード作成だ。Processingの実行画面上で表示させるには、自分でオリジナルのキーボードを作るしかなかった。さらに、キーボード上でクリックした文字を各マスに配置するというプログラムを作成することもかなり苦労した。結果として、無事ゲームのデモンストレーションを完成させることができた。また、作成した3個のゲームのデモンストレーションをAdobeのPremiere proで編集し、動画化した。中間発表までの課題だった、ゲームのデモンストレーション作成と動画作成は無事完成させることができた。今後の課題としては、Unityでのゲーム作成、作成したゲームの実装、動画編集技術の向上である。\\
 次にプロジェクト全体の役割としては、私はポスター班としての活動をしてきた。はじめのポスターの構成を考える段階では、Google スライドを使用して話し合いを行った。昨年のプロジェクトのポスターや様々な企業のポスターを参考にし、構成を考えた。構成が完成し、次にポスターに記載する各グループの概要や成果物の説明を考えた。先生方からのたくさんのご指摘を踏まえて、最終的には全員が納得する文章を考えることができた。ここまで、Google スライドを使用して作成していたのだが、イラストや文章の配置がうまくいかずAdobeのInDesignを使用して作成することになった。私は、InDesignの経験がなく、使用方法を勉強することから始めた。不慣れなツールを使用しながらではあったが、中間発表までに納得のいくポスターを完成させることができた。今後の課題としては、InDesignの技術向上とポスター作成に関する知識向上である。
\bunseki{田中龍仁}

\subsection{保土沢朋和}
グループBとして活動を始めた際に、新聞を活かしたゲームについて話し合った。その時に、モンスターを育成するゲームを提案した。このとき私は、とても面白いゲームができると思っていたが、新聞をうまく取り込めておらず、今回のプロジェクトとは少しずれていたため、没案となった。その際、グループ全体に伝えていたものと私が考えていたものが正しく伝わっておらず、没案になる際に自分の考えていたものを正しく伝えることになった。そのため、自分の考えているものを相手に共有するのは容易ではないことを学んだ。その後、先生の助言と話し合いの結果、地域的・歴史的辞書を用いた言葉遊びゲームを作ることになった。その間での話し合いでは意見が没案になったことを引きずってしまい、話し合いで積極的に意見を述べることができていなかった。ここはとても反省すべき点と思っている。\\
 中間発表に向けた制作物はWebを担当した。Webの視認性を高めるために、イラストやグラフを多めに使うことを意識した。ヘッダーの画像を制作し、画像カルーセルに合わせた画像を制作した。Web制作を通じて、視認性についての学習が深まったと思う。\\
 ゲーム班内での分担では機械学習班に振り分けられている。機械学習班は自然言語処理と文字認識の学習を課題としている。現段階で私は、自然言語処理の学習を行っており、10月ごろから始まる開発に向けて準備中である。自分の学習ペースは遅いため、現在からも学習しつつ夏休みからは学習ペースを上げて進めていく予定である。現在の自分の課題としては、自然言語処理と文字認識の知識が乏しく、チームに貢献できていないという点が挙げられるため、これらの学習を迅速に進めていきたい。
\bunseki{保土沢朋和}

\subsection{伊藤太一}
私が主に前期システム情報科学実習で行った活動としては、新聞ビッグデータにかかわる案出し、ポスター制作、グループ報告書の制作である。個人的に力を入れたのは案出しである。このプロジェクトは今年度から発足したことから、過去の成果物に囚われない目標設定が可能であったからである。過去の約130年分の新聞データを活用できる点に着目し、積極的に過去の新聞データを活用できる方法を模索した。明確なグループができる以前は、「新聞擬人化対戦」、「新聞を読んだユーザが何を思い浮かべているかを当てるツール」、「各新聞社の記事を分析し3Dデータ化することによって記事内容の偏りを体感できるツール」などの提案を行った。新聞ビッグデータを活用して言葉遊びゲームを制作するグループBに配属してからは、パズルゲーム、テトリスと新聞ビッグデータを結びつけるゲームを提案し、ゲーム集の1つとして制作することになった。\\
 中間発表にかかわるポスター制作場面においては、文章の背景と目的の文章を作成した。その中でグループのメンバと文章に一貫性があるか、矛盾はないか等の議論をした。また、ポスターデザインに関する情報を調べ、レイアウト、どのフォントを使用すべきか提案した。\\
 グループ報告書の制作場面では、本グループがどのような動機を持って活動するか、その根拠となる背景について調べ、執筆した。総務省が公開している「主なメディアの利用時間と行為者率」と日本新聞協会が公開している「新聞の発行部数と世帯数の推移」の2つのデータをもとに、若者の新聞離れという状況と、新聞業界自体が衰退しているという背景を数字による説明とグラフで可視化した。\\
 前期の個人的な反省点としては、2つある。1つ目は、自分の発言に自信を持てないことが多く、発言する機会を自分自身で減らしてしまったことである。今後このプロジェクトの発展のためには、どんな些細なことでも発言して意見交換することが重要である。2つ目に、ゲーム開発にかかわる技術習得を怠ってしまったことである。このグループで開発に使用するツールは主にUnityというゲームエンジンであるが、このツールは汎用性が高い一方で学ぶべき項目が多い。後期でいざ開発するとなったときに、ツールをある程度使いこなせていないと厳しい。後期で開発に困らないように、今からでも技術習得に向けて学習していく。
\bunseki{伊藤太一}

\subsection{小山内魁人}
主に背景や目的の設定について注力した。背景を調べる際にはより信用できる情報を集めることに気をつけた。また、周りが発信していた様々な情報から、何か新たな発見はないか気を配った。その中で主に3つの背景を調べることができた。1つ目として、新聞の購読者数についてである。新聞の購読者数は年々減っているのではないかという意見が多く、実際に正しいことが分かった。また、これを調べていると、年代別の情報にたどり着くことができ、若者の新聞についての現状を知ることが出来た。2つ目に新聞の信頼性についてである。今では、SNSなど様々、情報伝達方法があることに着目し、それらと新聞とを比較してどのような現状があるかを調べた。その結果、信頼性について、一般の人々が新聞に信頼をおいているということが分かった。3つ目に新聞の現状についてである。新聞は様々な面で問題があることを協力者である北海道新聞函館支社長の三浦辰治さんから情報を受け取っていた。その中で、ニュース砂漠という問題が起こっていることを知ることが出来た。それについて話し合いをしたことで、新聞には地域性という特徴があることに気づくことが出来た。これを通じてこれから活動する背景はどのようなものか、また、どのような方法を用いることで話しあいをスムーズに行うことができるかを学ぶことができた。\\
 目的の設定では、これらの背景をもとに自分たちがどのようなことを成果物を用いて行いたいのかを考える必要があった。はじめに、目的として、新聞の購読者数を増やすという目的を設定した、しかし、この目的は本当に達成可能なのかという指摘をうけて、その後、現在の目的へと変更した。これらから目的の設定の際には、自分たちの成果物が本当にしたいことは何なのかに気を付けて、達成可能であるのか、またそれが何につながるのかなどを考えて設定することを学べた。今後は、ゲーム制作班として活動していく。そのため、これから使用するゲーム開発ツールのUnityや機能の作成に必要なC\#をより深く学んでいく必要がある。
\bunseki{小山内魁人}

\subsection{日置竜輔}
私はグループBのリーダとして前期のシステム情報科学実習を行ってきた。意識して取り組んだことは、効率よく作業を進めていくために全員に仕事を振り分けることである。グループワークを行うにあたり、一人に負担がかかってしまうと作業効率が悪くなってしまうので、一人ひとりに何をしてもらうか常に役割を振るように意識した。その結果、プロジェクト内で決めた期日通りに事を運ばせることができた。私の活動内容としては、前期のシステム情報科学実習で案を練ることはもちろんだが、技術の向上についてもとても力を入れた。\\
 新聞ビッグデータを分析するには、新聞記事を文や単語、品詞ごとに分解する自然言語処理と呼ばれる手法を学習する必要がある。自然言語処理はライブラリが豊富なPythonを用いて学習した。私は、Pythonをある程度学習していたのでスムーズに自然言語処理の勉強に入ることができたが、同じ機械学習班のメンバにはPythonが初めてというメンバがいたので、基礎的な文法や知識を教える活動にも取り組んだ。その結果、私自身は新聞記事をMeCabと呼ばれるモジュールを使用し、品詞ごとに分解して、その出現頻度を可視化したグラフや、北海道に関連した言葉の入った新聞記事を抽出することに成功した。また、Pythonを教えたメンバも文法は理解出来た上に、自然言語処理の基礎を学習できたなどの成果が得られた。\\
 他にも中間発表のWebページで掲載するためのゲームのデモンストレーションを作成するにあたって、Processingを使用し作成した。学部1年の頃に使用していたのである程度は覚えていたが、再度使用すると忘れている部分もあり調べながら実装した。制限付き言葉遊びのデモンストレーションの作成を担当したが、カーソルが文字の上に乗っかることで色を出すホバーの機能などは作成するのにとても苦労した。私は、ゲーム開発班ではないので、機械学習が中心になっていく予定だが、ゲーム開発班が困っていたらサポートしていくつもりである。
後期の活動は実際にビッグデータを全て解析し、実際に言葉遊びゲームの中に分類したデータを渡す必要がある。したがって、さらに細かく解析する、ゲーム開発班の要望に合ったデータを渡すといった内容を後期では取り組んでいくつもりである。
\bunseki{日置竜輔}

\subsection{中川翔真}
このプロジェクトで新聞ビッグデータを見て最初に提案したことが新聞からデータを集め、お料理情報などをまとめることだった。年代やビッグデータを利用して移り変わりやレシピ集、さらには見やすく提供するものを想定していた。その後全員の案をまとめ、興味を持った分野についての話し合った結果、言葉に着目した。特に回文や名言制作など言葉に関することを追求し、回文探査について具体的な指針を考えた。回文の定義をどうするか、何文字以上の回文が見つかると面白味があるか、文章として成り立つものにするにはどうしたらよいかについて話し合った。次に大まかな3つの班に分かれ、その中で新聞データを調べた。そして、班員と言葉遊びについて話し合った。結果、具体的な成果物が見えず、私が属していた班は解体されゲーム班に移籍した。そこで、すごろくとミニゲームという案に加えて、私が元々話していた言葉遊びという案を持ち込んだ。ゲーム班で制作するにあたって背景を調べたところ、The New York Timesがクロスワードパズルで成功している例があり、言葉遊びを中心とするゲームを制作することで決定した。クロスワードパズルやヒットアンドブローなど様々なアプローチから言葉遊びゲームを展開していく中で、私は言葉の読まれ方の変化に目を付けた。シルエットから名前を当てるゲームを提案していたが、アプローチを変え次のようなゲームを提案した。それは、鮭にはシャケやホッチャレなど年代や地域によって様々な呼ばれ方があることを利用したゲームである。\\
 また、中間発表会に向けてポスターの制作を行った。はじめにポスターの内容に関して今まで作業していたスライドの文章を添削した。そのなかで当初話し合っていた私が提案したゲームのタイトル名であるNew Super News paperが採用されそうになっていた。その後は作業が被らないよう、積極的に分担を行い効率よく進めた。そのため、レイアウトを考えるメンバと、文章を考えるメンバに分かれて作業を行った。また、ツールで誤った部分を自分の考えをもとに訂正した。中間発表会では書記をつとめており、質疑応答について正確にメモをとった。今後に活かすために書き方も工夫した。中間報告書の作成では主に背景を担当し、数値を用いてグラフを作成した。後期ではゲーム制作が主な活動になるので、プロジェクトについていけるよう頑張っていきたい。
\bunseki{中川翔真}

\subsection{金澤快飛}
システム情報科学実習の最初の活動として、新聞の特徴やそれに関わる現状の問題を個人で調査した。今まで私は新聞に触れる機会があまり多くはなかった。そのため、この調査では、なぜ新聞が読まれているのか、またなぜ新聞が必要なのかを知ることができ、自分にとって良い機会となった。その後の活動では、新聞記事データを使用して何ができるのかそれぞれがアイデアを考え、発表形式で意見交換を行なった。前期の活動では、特にこのアイデア出し作業にプロジェクトメンバ全員で力を注いだ。活動が進みアイデアが集約していくと本プロジェクトのテーマとして、メディア班とゲーム班に二分されるようになった。私は過去に画像処理について学習していた経験があり、新たなAI技術を習得したいという経緯のもと、ゲーム班内の機械学習班を担当することを選択した。\\
 機械学習班では、新聞ビッグデータの分析、画像データ内文字のテキスト変換、地域性・歴史性のある辞書の作成が課題として挙げられた。これらの課題を達成するためには、プログラミング能力の向上、自然言語処理や機械学習の知識の習得が必要になる。そこで、前期の活動としては主に、本プロジェクトで使用するPython言語や自然言語処理についての学習を行なった。Python言語の学習は、競技プログラミングサイト内の教材を解きながら進めていき、Python言語の特徴や基礎構文について理解することができた。自然言語処理については、実際にプログラムを書きコードを動かしながら自然言語処理を学習することのできる「言語処理100本ノック 2020」[6]というサイトを利用しながら学習を進めていった。このサイトでは、自然言語処理の知識が習得できるだけでなく、プログラミング能力を向上させることができ、効率的学習が可能であることから、今後も続けていこうと予定している。中間発表では、発表資料の素材として、自分が提案した企画と同メンバ一人分のゲームUIの作成をWindows標準ソフトであるペイント3Dを用いて行なった。今後のゲームへの導入を考え、3DモデリングソフトBlenderでのモデリングを想定していたが、期限の制約があり、取り組むことができなかった。そのため、夏休み期間を利用してのモデリング技術の習得も視野に入れている。前期の活動が終了した現在、今後の流れとしては新聞ビッグデータから画像内文字のテキスト変換を行なったのち、テキストデータが詰まったデータベースを作成しようと計画している。そのため、実現に必要な知識の習得、具体的な手法の検討をメンバ間での情報共有・会議などを通して行なっていくつもりだ。
\bunseki{金澤快飛}
\section{今後の予定}
中間発表を終えて、今後改善すべき課題が見つかった。グループBでは、改善すべき課題を踏まえて、下記の予定で活動を行っていく。\\
 8月、9月では、グループメンバで集まり、背景の調査や目的を再確認する。現在調べた背景とグループBのつながりが薄いため、背景の深掘りを行いグループBの目的とのつながりを明確にする。具体的には、新聞の購読者数減少が社会的に問題である理由や、ゲームを用いることの有用性について調べる。また、機械学習班は画像の文字をテキストに変換するための文字認識の知識や、辞書を作成するために自然言語処理や機械学習の勉強を行う。ゲーム開発班は、ゲームを開発するにあたりUnityやC\#の勉強を行う。具体的には、テキストの扱い方とUIの強化について勉強する。これらを学習する手法として、Webや論文などの参考資料を実際の新聞データに活用して勉強していく。\\
 10月、11月では、それぞれの班に分かれて開発を行う。機械学習班では、自然言語処理を用いて、地域的、歴史的な辞書の作成を行う。また、画像の文字をテキストに変換するために、文字認識を行い開発を進めていく予定である。ゲーム開発班では、機械学習班が作成した辞書を利用し、Unityを用いて言葉遊びゲームの開発を進めていく。その際、Git/Githubを用いた開発を行う予定である。自分たちの成果物について、客観的な意見や感想を集めるために評価実験を行う。\\
 12月では、期末報告書の作成を行う。その際に、発表用スライドの作成など、中間発表で参考になった意見や中間報告書での改善点を活用する。
\bunseki{小山内魁人}