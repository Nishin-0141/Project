\chapter{中間発表}
第5章では、中間発表の準備や発表、講評について紹介する。中間発表で使用するポスターとWebの制作をする際に工夫したことやグループB全体の反省点などをまとめている。
\section{準備・過程}
% 担当:日置竜輔
中間発表では主にポスターとWebの制作について取り組んだ。
ポスター制作に関しては、これまでの活動を説明するだけではなく、より詳細な情報を持っているWebへ誘導する必要があった。初期案では、ゲームのデモンストレーション画面とゲーム全体のUI画面を使用することを考えていた。しかし、ゲーム全体のUI画面は伝えられる情報量が少ないので、削減した。制作途中で、画像を使用する際に元の画像から縦横比を変えたものを使用していたが、元の情報を適切に伝えたり、画像の見栄えを維持するため、元の画像の比率のまま使用することにした。また、行間を広く取った結果、文章量が少なく、情報が少ないものとなってしまった。そのため、行間を狭くし、文章量を多くすることで情報を増やした。
Webの制作に関しては、ポスターと違い、指定された枠組みがないため、出来る限り細かく書いて詳細まで理解してもらえるように取り組んだ。さらに、本文が短く、見出しが画面を埋めてしまうことのないように、イラストや図を用いることで、UIも良くなるように工夫した。また、ゲームのデモンストレーション動画をYouTubeに投稿し、Web内に埋め込むことで具体的にどのようなものを制作する予定なのかを伝える工夫をした。
\bunseki{日置竜輔}

\section{発表}
% 担当:日置竜輔
発表は7月9日金曜日の15:00 - 18:00に Zoom を用いてオンラインで行われた。まず、15:00 - 16:00の1時間を使用し事前に制作したポスター・Webの2つを聴講者に見てもらった後、前半と後半でそれぞれ3つの時間に分かれて質疑応答の時間が設けられた。発表では、事前に司会の進行役や質問の内容に対する回答者を分担していたため、潤滑に進行する事ができた。しかし、想定外の観点から質問が来て答える際に時間をかけてしまい、準備不足を感じる一面もあった。他のプロジェクトでは、質疑応答の時間が終了した後、評価URLを Zoom のチャット欄に貼っていたり、Zoom のカメラの背景をプロジェクトメンバ内で統一していた等、参考に出来る点が多くあった。
\bunseki{日置竜輔}

\newpage
\section{講評}
中間発表終了後に、聴講者の意見が集約したエクセルでのフィードバックを回収した。回収した件数は、37件であり教授が5件、学生が32件であった。また、発表技術の評価、発表内容の評価ともに平均点は7.8378378....(=290/37)であった。発表内容の評価の中には、「新聞に着目しすぎてビッグデータとの関連性が見られない」といった内容の意見が出ていた。今までグループBおよびプロジェクトとして、新聞との関連性のみを考えていた。グループBは、ビッグデータを多く利用する機会があるため、ビッグデータとの関連性についても今後話し合うことが必要であるとフィードバックを通して意見共有がでた。また、発表技術に関して「質疑時間に詳細に説明していた内容を、公開しているサイトにもう少し記載した方が良い」といった内容の意見があった。他にも質疑応答で、「Webを使用して説明するよりも発表用のスライドを作成して発表した方が良い」といった内容の意見も見られたため、発表で工夫をしていくことが必要である。\\
 講評に関して、課題になるような意見を紹介してきたが、最終成果発表に活かせる意見もあった。例えば、「質問が無い際にグループの説明をした点がよかった」といった内容の意見である。この点に関して、良いと感じたメンバが多数いたため、最終成果発表でも行う予定である。また、「発表者が質問しやすい環境を作ってくれてよかった」といった内容の意見もあった。このような意見を参考にしりする。\\
 グループBでの発表の客観的評価は、5段階中で4段階だとメンバと話し合い決定した。その際に、評価の理由になった項目について記述していく。\\
 はじめに、目的に関してである。目的に関しては、背景との繋がりが弱くまだ協議していくべきであると考えた。 次に、今後の計画の具体性に関してである。今後の計画では、ゲーム制作班ではゲームを、機械学習班ではデータ処理をやることが既に決まっている。そのため、計画の具体性はあるといえる。また、項目のひとつである表現力に関しては、ポスターとWebで十分に発揮することができたと考えた。ポスターでは、レイアウトを最優先で作成し、細かい調整も複数回行った。Webでは、デモンストレーション動画を入れることにより、説明のしやすいものにした。また、中間発表で役割分担をすることにより、適切な表現を行うことができた。最後に、チームワークに関してである。チームワークについては、グループメンバ間でタスクを振り分け共有することにより、1人だけ何もしない状況を回避することができた。また、授業時間外でも集まることにより授業内での進行を円滑にすることができた。最終成果発表では、以上の項目を意識するつもりである。
\bunseki{金澤快飛}