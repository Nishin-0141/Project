\chapter{はじめに}
%該当分野の従来の状況、問題点、本プロジェクトで設定した課題、実施した解決策、及び成果を簡潔に記述する。

\section{プロジェクトの概要}
% 担当:
%プロジェクトの分野の状況や、目的、背景を記述する。 
近年、インターネットメディアの台頭があり、新聞記事は高い信頼性のある情報が含まれているにもかかわらず、人々に十分に届いていない。また、SNSやスマートフォンの普及などを背景に、デジタルデータは急速に増加している。こうした多様かつ大量のデータを効果的に活用することで、新たな価値を生み出すことが可能となった。そこで、本プロジェクトは、北海道新聞社の協力のもと、過去130年ほどの新聞記事データを活用し、新聞記事をあまり読むことのない人に向け、アクセスする機会を増やすことを目的とした。
\bunseki{岩上慎之介}

\section{グループBの概要}
% 担当:
グループBでは、新聞の地域的・歴史的な面に焦点を当て、新聞の記事データから得られる言葉をもとに「クロスワードパズル」や「ヒットアンドブロー」などの言葉遊びゲームを制作する。また、地方紙が減少し、ニュース砂漠と呼ばれる現象が起こっており、地元の人々が、適切に身近な情報を手に入れる手段が減少している。The New York Times では電子版の新聞にクロスワードパズルを用いることで、新聞の購読者数を安定させた。そこで、言葉遊びゲームを通して北海道という地域や時代特有の言葉に触れてもらうことを目的として活動を行った。
\bunseki{岩上慎之介}