\chapter{はじめに}
%該当分野の従来の状況、問題点、本プロジェクトで設定した課題、実施した解決策、及び成果を簡潔に記述する。

\section{プロジェクトの概要}
% 担当:
%プロジェクトの分野の状況や、目的、背景を記述する。 
近年、インターネットメディアの台頭があり、新聞記事は高い信頼性のある情報が含まれているにもかかわらず、人々に十分に届いていない。また、SNSやスマートフォンの普及などを背景に、デジタルデータは急速に増加している。こうした多様かつ大量のデータを効果的に活用することで、新たな価値を生み出すことが可能となった。そこで、本プロジェクトは、北海道新聞社の協力のもと、過去130年ほどの新聞記事データを活用し、新聞記事をあまり読むことのない人に向け、アクセスする機会を増やすことを目的とした。
\bunseki{岩上慎之介}

\section{グループBの概要}
% 担当:
グループB では、新聞の地域的・歴史的な面に焦点を当て、新聞の記事データから得られる言葉をもとに「クロスワードパズル」や「ヒットアンドブロー」などの言葉遊びゲームを制作する。また、地方紙が減少しニュース砂漠と呼ばれる現象が起こっており、地元の人々が適切に身近な情報を手に入れる手段が減少している。The New York Times では電子版の新聞にクロスワードパズルを用いることで、新聞の購読者数を安定させた。そこから発想を受け、言葉遊びゲームを制作し、新聞が持っている地域性や時代特有の言葉に触れてもらい、新聞の持つ地域性を再確認させることを目的として活動を行った。
\bunseki{岩上慎之介}